% Chapter Template

\chapter{Conclusion and Future Work} % Main chapter title

\label{Chapter7} % Change X to a consecutive number; for referencing this chapter elsewhere, use \ref{ChapterX}

\lhead{Chapter 7. \emph{Conclusion and Future Work}} % Change X to a consecutive number; this is for the header on each page - perhaps a shortened title

%----------------------------------------------------------------------------------------
%	SECTION 1
%----------------------------------------------------------------------------------------

\section{Conclusions}
The master thesis presents the use of new technologies, such as smart watches and beacons in context of mobile phone applications. We proposed and implemented two methods 
of accessing smartwatch features and provided an extensive power analysis of using them in real life scenarios. As results of our research we demonstrated that SWAN is able to run
on any Android Wear device. SWAN running on the smartwatch was able to surpass simple data collection in any test scenario in power efficiency,
providing superior power savings on both phone and watch.

The added support for beacon based sensors allowed us to provide specifications for data formats which can be used as guidelines for adding support of other sensor types. We proposed two
specification drafts and we opted for implementing the option which have the best compatibility with the existing implementation.

We noticed that beacon market suffer from very high fragmentation despite the effort from Google and Apple to offer a standard. However, we managed to overcome this obstacle by using
a single, open-source library for parsing. Moreover, we proved that even a closed source format such as Estimote Nearable can be easily analyzed and supported without the use of the proprietary development kit .

%-----------------------------------
%	SUBSECTION 1
%-----------------------------------
\section{Future Work}
Our power saving tests revealed inconsistent run times for the same amount of requested values and constant delay. For some test cases we noticed a run time value twice as high as expected one.
Long run times can also translate in poor power efficiency. We were able to identify few problems with SWAN delay enforcing implementation, but the problem should be further investigated.

We were able to prove that multiple sensors of the same type can be integrated with SWAN, by integrating support for multiple beacons and implementing few test sensors to retrieve values.
Unfortunately, we were only able to validate the sensors with Swan Monitor\cite{swan_monitor} application. Building a real proof-of-concept application that use beacons to present real use cases should be done in near future.

