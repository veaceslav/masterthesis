% Chapter Template

\chapter{Introduction} % Main chapter title

\label{Chapter1} % Change X to a consecutive number; for referencing this chapter elsewhere, use \ref{ChapterX}

\lhead{Chapter 1. \emph{Introduction}} % Change X to a consecutive number; this is for the header on each page - perhaps a shortened title

%----------------------------------------------------------------------------------------
%	SECTION 1
%----------------------------------------------------------------------------------------
With the smartphone user base reaching almost two billion users, new technologies arise to take advantage of our pocket device which also packs a considerable amount of processing power.
Smartwatches who pair with our smartphones aim to replace the regular watch, beacons aim to replace WI-FI location support. 
However, we must not forget that limited battery capacity is still a problem even after so many years of development. The power consumption is even more limited on the smartwatch, because of
its small form factor.
With Android Ecosystem being mature enough, there are little room for new application ideas. Having reliable access to new features and taking advantage of the new Bluetooth Low Energy standard
enable us to come with new, innovative usage scenarios. We try to take advantage of the new technologies by extending SWAN application.

The master thesis is focused on the following research questions:

\begin{itemize}
 \item What is the best method of acquiring data from a sensor located on the smartwatch? Can SWAN operate on the smartwatch, assuming that 
 \item What changes should be operated on expression based SWAN, to add support for multiple sensors of the same type located on beacons?
 \item How can we take full advantage of Bluetooth Low Energy Beacons? How do we seamlessly integrate multiple beacon frame formats without increasing the program's complexity?
 \item What are the power consumption on the watch and phone, when SWAN is actively retrieving data from smartwatch?
\end{itemize}

The master thesis is structured as follows. In Chapter 2 we will describe the related work to out project, chapter 3 covers only essential details about the SWAN project and chapter 4 is dedicated to 
data format changes that we operated on SWAN expression to allow support of multiple sensors. Chapter 5 describes all implementation details related to scanning beacons and communication
with smartwatch and Chapter 6 offer a detailed analysis of the power consumption for two implemented methods of accessing smartwatch sensors. Chapter 7 concludes the master thesis and describe 
the future work.
