% Chapter Template

\chapter{SWAN Data Specifications} % Main chapter title

\label{Chapter4} % Change X to a consecutive number; for referencing this chapter elsewhere, use \ref{ChapterX}

\lhead{Chapter X. \emph{Chapter Title Here}} % Change X to a consecutive number; this is for the header on each page - perhaps a shortened title

%----------------------------------------------------------------------------------------
%	SECTION 1
%----------------------------------------------------------------------------------------

\section{Introduction}
Initially SWAN was designed to take a single value from the sensor data.
The value should be a primitive type such as Integer, Float, Double etc.
This constraint applies because SWAN evaluation engine allows the programmer to perform MIN, MAX and AVERAGE operations on given sensor data.
Also, History reduction is important when using the single value.

By adding multiple sensors, it became clear that taking a single value is not scalable enough. 
The further changes were delayed by adding multiple value paths to the sensor implementation. 
A clear example was accelerometer data: You will probably need the values for all 3 axis: x, y, z.
And you need to register 3 expressions for each value path.
The limitation can no longer be avoided after implementing Beacon Sensors, because now we need to add a full array of beacon identifiers under the same timestamp,
the key-value to provide individual temperature data or even key-array of values if we want the accelerometer data from all beacons.

%-----------------------------------
%	SUBSECTION 1
%-----------------------------------
\section{Proposed Solutions}
All the proposed solutions should met the following requirements:
\begin{itemize}
 \item Support multiple sensors of the same type(beacon sensor)
 \item Support multiple values for each sensor
 \item Provide scheme for single sensor single value data
 \item Provide implementation for Evaluation Engine Application
\end{itemize}

\subsection{Using a Mapping between keys and values to get all the data}
The format of data for the following function calls(value parameter):
\begin{lstlisting}[language=Java]
 AbstractSwanSensor.putValueTrimSize(final String valuePath,
					final String id,
					final long now,
					final Object value):
\end{lstlisting}

In order to satisfy the new requirements we propose the new data format to be encoded as: \begin{verbatim} Map<ValuePath, Map<sensorID,  ArrayList<Object>> \end{verbatim}

where:
\begin{itemize}
 \item ValuePath  = sensor valuepaths, ex Acceleromter x, y, z, total
 \item sensorID  =  self or unique Sensor Identifier(BeaconID)
\end{itemize}

We identify four sensor formats and recommend the following representation for each type of sensor:
\begin{itemize}
  \item Single Sensor Single Value(sensor name = stepcounter):
  \begin{itemize}
    \item \begin{verbatim} Example: Map: { steps={self=[13434545 ] }} \end{verbatim}                                                              
  \end{itemize}
 
  \item  Single Sensor, Multiple Values(sensor name =accelerometer):
  \begin{itemize}
    \item Key - name of the sensor
    \item Array of values - multiple values in the same order
    \item \begin{verbatim} Example: Map: { x={self=[0.5]},
                 y={self=[0.6]}, z={self=[-0.5]} } \end{verbatim}  
  \end{itemize}

 
  \item Multiple Sensor, Single Value(Beacon Distance):
  \begin{itemize}
    \item Key - the sensor identifier(id)
    \item Array of values - single value in the array
    \item  \begin{verbatim}  Example: Map: {distance={beaconID1=[1.5],
                 beaconID2=[0.5], beaconID3=[0.2]}} \end{verbatim}  
  \end{itemize}

  \item  Multiple Sensor, Multiple Values(sensor name = Beacon Accelerometer):
  \begin{itemize}
    \item Key - the sensor identifier(id), should be unique
    \item Array of values - multiple values in the same order
    \item  \begin{verbatim}  Example: Map:{x ={beaconID1=[-0.5], beaconID2=[-0.1],
                y = {beaconID1=[ 0.2], beaconID2=[-0.4]}} \end{verbatim}  
 \end{itemize}


\end{itemize}




%-----------------------------------
%	SUBSECTION 2
%-----------------------------------

\subsection{Subsection 2}

%----------------------------------------------------------------------------------------
%	SECTION 2
%----------------------------------------------------------------------------------------

\section{Implemented Approach}
