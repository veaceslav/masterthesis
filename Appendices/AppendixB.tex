% Appendix Template

\chapter{Merging SWAN Repositories} % Main appendix title

\label{AppendixB} % Change X to a consecutive letter; for referencing this appendix elsewhere, use \ref{AppendixX}

\lhead{Appendix B. \emph{Merging SWAN Repositories}} % Change X to a consecutive letter; this is for the header on each page - perhaps a shortened title
At the beginning of the master thesis project, the main repository of SWAN was holding the stable version of it, which was being used in production 
by a  company. Our SWAN project has a big team, each member work on different part of the SWAN or SWAN related functionality.
To avoid any breaking changes to the current SWAN version, we decided to make a different branch and put all my smartwatch related changes in it.
When we was done with implementing the first prototype of SWAN WEAR sensors, other PhD students' projects: SWAN Plus\cite{swan_plus} and SWAN-Fly\cite{swan_fly} were also ready.
Unfortunately, the commits of the projects stated above were in different repositories so merging them into master branch was proven to be difficult.
We followed this procedure when merging the SWAN:
\begin{itemize}
 \item Create the release branch with the old, stable SWAN
 \item Create development branches for SWAN Plus and SWAN-Fly
 \item Export commits from SWAN Plus and SWAN-Fly as patches
 \item Apply the patches to each development branch
 \item Merge Smartwatch code into Master
 \item Merge SWAN Plus into master and resolve conflicts
 \item Merge SWAN-Fly into master and resolve conflicts
\end{itemize}

The most challenging part was to apply patches from the two different repositories. Merging commits were the reason why some history was 
lost, and patches were hard to apply. This also proves why you should always use the rebase option instead of merging option when you push commits
to the remote. Atlassian article\cite{atlassian_merge_rebase} explains the main advantage of rebasing over using merging commits.