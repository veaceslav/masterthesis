%%%%%%%%%%%%%%%%%%%%%%%%%%%%%%%%%%%%%%%%%
% Oliver Lemon made minor edits (jan 2015)  to : 
% Masters/Doctoral Thesis 
% LaTeX Template
% Version 1.43 (17/5/14)
%
% This template has been downloaded from:
% http://www.LaTeXTemplates.com
%
% Original authors:
% Steven Gunn 
% http://users.ecs.soton.ac.uk/srg/softwaretools/document/templates/
% and
% Sunil Patel
% http://www.sunilpatel.co.uk/thesis-template/
%
% License:
% CC BY-NC-SA 3.0 (http://creativecommons.org/licenses/by-nc-sa/3.0/)
%
% Note:
% Make sure to edit document variables in the Thesis.cls file
%
%%%%%%%%%%%%%%%%%%%%%%%%%%%%%%%%%%%%%%%%%

%----------------------------------------------------------------------------------------
%	PACKAGES AND OTHER DOCUMENT CONFIGURATIONS
%----------------------------------------------------------------------------------------

\documentclass[11pt, oneside]{Thesis} % The default font size and one-sided printing (no margin offsets)

 \usepackage{url}
\graphicspath{{Pictures/}} % Specifies the directory where pictures are stored
%\usepackage[square, comma, sort&compress]{natbib} % Use the natbib reference package - read up on this to edit the reference style; if you want text (e.g. Smith et al., 2012) for the in-text references (instead of numbers), remove 'numbers' 
\hypersetup{urlcolor=blue, colorlinks=true} % Colors hyperlinks in blue - change to black if annoying
\PassOptionsToPackage{hyphens}{url}\usepackage{hyperref}


\usepackage{color}
\usepackage{multirow}
\definecolor{javared}{rgb}{0.6,0,0} % for strings
\definecolor{javagreen}{rgb}{0.25,0.5,0.35} % comments
\definecolor{javapurple}{rgb}{0.5,0,0.35} % keywords
\definecolor{javadocblue}{rgb}{0.25,0.35,0.75} % javadoc
 
\lstset{language=Java,
basicstyle=\ttfamily,
breaklines=true,
keywordstyle=\color{javapurple}\bfseries,
stringstyle=\color{javared},
commentstyle=\color{javagreen},
morecomment=[s][\color{javadocblue}]{/**}{*/},
numbers=left,
numberstyle=\tiny\color{black},
stepnumber=2,
numbersep=10pt,
tabsize=4,
showspaces=false,
showstringspaces=false}

\title{HWU CS Masters thesis template} % BUT you should use use " \title{\ttitle} " here instead to define the thesis title ! 
% \ttitle is defined in the file Thesis.cls 

\begin{document}

\frontmatter % Use roman page numbering style (i, ii, iii, iv...) for the pre-content pages

\setstretch{1.3} % Line spacing of 1.3

% Define the page headers using the FancyHdr package and set up for one-sided printing
\fancyhead{} % Clears all page headers and footers
\rhead{\thepage} % Sets the right side header to show the page number
\lhead{} % Clears the left side page header

\pagestyle{fancy} % Finally, use the "fancy" page style to implement the FancyHdr headers

\newcommand{\HRule}{\rule{\linewidth}{0.5mm}} % New command to make the lines in the title page

% PDF meta-data
\hypersetup{pdftitle={\ttitle}}
\hypersetup{pdfsubject=\subjectname}
\hypersetup{pdfauthor=\authornames}
\hypersetup{pdfkeywords=\keywordnames}

%----------------------------------------------------------------------------------------
%	TITLE PAGE
%----------------------------------------------------------------------------------------

\begin{titlepage}
\begin{center}

\textsc{\LARGE \univname}\\[1.5cm] % University name
\textsc{\Large Masters Thesis}\\[0.5cm] % Thesis type

\HRule \\[0.4cm] % Horizontal line
{\huge \bfseries \ttitle}\\[0.4cm] % Thesis title
\HRule \\[1.5cm] % Horizontal line
 
\begin{minipage}{0.4\textwidth}
\begin{flushleft} \large
\emph{Author:}\\
\href{http://www.linkedin.com/in/veaceslav-munteanu-4370a063}{\authornames} % Author name - remove the \href bracket to remove the link
\end{flushleft}
\end{minipage}
\begin{minipage}{0.4\textwidth}
\begin{flushright} \large
\emph{Supervisor:} \\
\href{http://www.cs.vu.nl/~bal/}{\supname} % Supervisor name - remove the \href bracket to remove the link  
\end{flushright}
\end{minipage}\\[3cm]
 
\large \textit{A thesis submitted in fulfilment of the requirements\\ for the degree of \degreename}\\[0.3cm] % University requirement text
\textit{in the}\\[0.4cm]
%\groupname\\

\deptname\\[2cm] % Research group name and department name
 
{\large \today}\\[1cm] % Date
\includegraphics[width=6cm]{./Figures/VUlogo.png} % University/department logo - uncomment to place it
 
\vfill
\end{center}

\end{titlepage}



%----------------------------------------------------------------------------------------
%	ABSTRACT PAGE
%----------------------------------------------------------------------------------------

\addtotoc{Abstract} % Add the "Abstract" page entry to the Contents

%\abstract{\addtocontents{toc}{\vspace{1em}} % Add a gap in the Contents, for aesthetics

 {\huge{\textit{Abstract}} \par}{\addtocontents{toc}{\vspace{1em}} 

A typical mobile phone applications will most likely want to take better advantage of the sensors embedded in the phone.
It is easy to access data about sensors on an Android device but things start to be difficult when we want good battery savings.
The best case scenario we will want to delegate all the troubles of having efficient implementation to a library, or a middleware project, such as SWAN.
SWAN offer a simple and intuitive way to get data from available sensors, just by registering a simple expression.
To make things even simpler, it offer a  configuration activity for the given sensor, so that developers can learn faster what options are available to them and how to get the best expressions for the needs. 
 The sensors in the phone offer great flexibility but with proximity sensors it becomes increasingly difficult to preserve the original design.
 For example applications want to take advantage from sensors located in Bluetooth connected smartwatch or nearby beacons . 
The functionality offered by SWAN extends beyond the simple sensor data retrieval. 
Advanced scenarios include waking up the application when the sensor value exceed a certain threshold. This allow applications to be closed and only wake up only when a certain event occurs.
 We will further discuss about smartwatch and beacon sensor implementation in SWAN. 
The most important topics covered are the  application design solutions, data format specifications and extensive information
about power efficiency of different data retrieval methods on smartwatch.  



%The page is kept centered vertically so can expand into the blank space above the title too\ldots
%

\clearpage % Start a new page

%----------------------------------------------------------------------------------------
%	ACKNOWLEDGEMENTS
%----------------------------------------------------------------------------------------

\setstretch{1.3} % Reset the line-spacing to 1.3 for body text (if it has changed)

\acknowledgements{\addtocontents{toc}{\vspace{1em}} % Add a gap in the Contents, for aesthetics

The acknowledgements and the people to thank go here, don't forget to include your project advisor :)  
}
\clearpage % Start a new page

%----------------------------------------------------------------------------------------
%	LIST OF CONTENTS/FIGURES/TABLES PAGES
%----------------------------------------------------------------------------------------

\pagestyle{fancy} % The page style headers have been "empty" all this time, now use the "fancy" headers as defined before to bring them back

\lhead{\emph{Contents}} % Set the left side page header to "Contents"
\tableofcontents % Write out the Table of Contents

\lhead{\emph{List of Figures}} % Set the left side page header to "List of Figures"
\listoffigures % Write out the List of Figures

\lhead{\emph{List of Tables}} % Set the left side page header to "List of Tables"
\listoftables % Write out the List of Tables

%----------------------------------------------------------------------------------------
%	ABBREVIATIONS
%----------------------------------------------------------------------------------------

\clearpage % Start a new page

% \setstretch{1.5} % Set the line spacing to 1.5, this makes the following tables easier to read
% 
% \lhead{\emph{Abbreviations}} % Set the left side page header to "Abbreviations"
% \listofsymbols{ll} % Include a list of Abbreviations (a table of two columns)
% {
% \textbf{LAH} & \textbf{L}ist \textbf{A}bbreviations \textbf{H}ere \\
% %\textbf{Acronym} & \textbf{W}hat (it) \textbf{S}tands \textbf{F}or \\
% }

%----------------------------------------------------------------------------------------
%	PHYSICAL CONSTANTS/OTHER DEFINITIONS
%----------------------------------------------------------------------------------------

%\clearpage % Start a new page

%\lhead{\emph{Physical Constants}} % Set the left side page header to "Physical Constants"

%\listofconstants{lrcl} % Include a list of Physical Constants (a four column table)
%{
%Speed of Light & $c$ & $=$ & $2.997\ 924\ 58\times10^{8}\ \mbox{ms}^{-\mbox{s}}$ (exact)\\

%% Constant Name & Symbol & = & Constant Value (with units) \\
%}

%----------------------------------------------------------------------------------------
%	SYMBOLS
%----------------------------------------------------------------------------------------

% \clearpage % Start a new page
% 
% \lhead{\emph{Symbols}} % Set the left side page header to "Symbols"
% 
% \listofnomenclature{lll} % Include a list of Symbols (a three column table)
% {
% $a$ & distance & m \\
% $P$ & power & W (Js$^{-1}$) \\
% % Symbol & Name & Unit \\
% 
% & & \\ % Gap to separate the Roman symbols from the Greek
% 
% $\omega$ & angular frequency & rads$^{-1}$ \\
% % Symbol & Name & Unit \\
% }


%----------------------------------------------------------------------------------------
%	THESIS CONTENT - CHAPTERS
%----------------------------------------------------------------------------------------

\mainmatter % Begin numeric (1,2,3...) page numbering

\pagestyle{fancy} % Return the page headers back to the "fancy" style

% Include the chapters of the thesis as separate files from the Chapters folder
% Uncomment the lines as you write the chapters

% Chapter Template

\chapter{Introduction} % Main chapter title

\label{Chapter1} % Change X to a consecutive number; for referencing this chapter elsewhere, use \ref{ChapterX}

\lhead{Chapter 1. \emph{Introduction}} % Change X to a consecutive number; this is for the header on each page - perhaps a shortened title

%----------------------------------------------------------------------------------------
%	SECTION 1
%----------------------------------------------------------------------------------------
With the smartphone user base reaching almost two billion users, new technologies arise to take advantage of our pocket device, which also packs a considerable amount of processing power.
Smartwatches that pair with our smartphones aim to replace the regular watch, beacons aim to replace WI-FI location support. 
However, we must not forget that limited battery capacity is still a problem even after so many years of development. The available power is even more limited on the smartwatch, because of
its small form factor.
Having reliable access to new features and taking advantage of the new Bluetooth Low Energy\cite{bt_low} standard
enable us to come with new, innovative usage scenarios, such as measuring heart rate of elderly care using the smartwatch or deploying beacons in context of smart homes.
The SWAN framework aims to facilitate the access to sensors available on the phone.
We try to take advantage of the new technologies by extending the SWAN framework.

The overall goal of the project was to integrate a variety of new Bluetooth Low Energy devices into SWAN,
including beacons and smartwatches. We encountered several challenging problems that we study in more detail,
including the following questions:

\begin{itemize}
 \item What is the best method of acquiring data from a sensor located on the smartwatch? Can SWAN operate on the smartwatch, assuming that we can only keep its core functionality?
 \item What are the power consumption on the watch and  the  phone, when SWAN is running on both the phone and the watch?
 \item What changes should be applied to expression based SWAN, to add support for multiple sensors of the same type located on beacons?
 \item How can we access all data on the Bluetooth Low Energy Beacons? How do we seamlessly integrate multiple beacon frame formats without increasing the program's complexity?
\end{itemize}

The master thesis is structured as follows. In Chapter 2 we will describe the work related to our project, Chapter 3 covers  essential details about the SWAN project and Chapter 4 is dedicated to 
beacon implementation and data format changes that we applied on SWAN  expression. Chapter 5 describes all implementation details related to  communication
with a smartwatch and Chapter 6 offers a detailed analysis of the power consumption for two implemented methods of accessing smartwatch sensors. Chapter 7 concludes the master thesis and describes 
the future work.

% Chapter Template

\chapter{Related Work} % Main chapter title

\label{Chapter2} % Change X to a consecutive number; for referencing this chapter elsewhere, use \ref{ChapterX}

\lhead{Chapter 2. \emph{Related Work}} % Change X to a consecutive number; this is for the header on each page - perhaps a shortened title

%----------------------------------------------------------------------------------------
%	SECTION 1
%----------------------------------------------------------------------------------------

\section{General}
For our implementation of smartwatch sensors and bluetooth beacon sensors we used standard Android API to send scan and receive information.
There are other ways of communicating with the smartwatch, for example using Beetle\cite{beetle_mobisys16} service to communicate with smartwatch.
Using Beetle it is also possible to retrieve data from heart rate monitor in real time, in a similar way of how our SWAN service on the watch works.

The implementation on the SWAN service on the smartwatch does not have notable differences compared to a regular android phone application.
An extensive description of the operating system on the watch is described in Understanding Characteristics of Android Wear\cite{android_wear_char}.
In comparison with our power consumption experiment, the paper's experiment was performed  with power monitors attached 
directly to hardware, which gives higher accuracy.

\section{Swan Plus}
SWAN Plus project, which aims to offer expression evaluation on nearby devices, is closely related to our work on adding support for smartwatches and Bluetooth devices.
Our implementation of sending SWAN expressions on the watch was coded on top of already existing mechanism of sending expressions to nearby devices and forwarding results back to the expression
registrar. On the other hand, SWAN Plus focuses on multiple devices and uses a more close-to-metal approach of scanning and discovering the Bluetooth devices.

 
% Chapter Template

\chapter{Background} % Main chapter title

\label{Chapter3} % Change X to a consecutive number; for referencing this chapter elsewhere, use \ref{ChapterX}

\lhead{Chapter 3. \emph{Background}} % Change X to a consecutive number; this is for the header on each page - perhaps a shortened title

%----------------------------------------------------------------------------------------
%	SECTION 1
%----------------------------------------------------------------------------------------
In this chapter, we provide a brief description of the SWAN framework.
The first subsection will focus only on key features of SWAN, relevant to our work.
The second subsection will briefly describe few Beacon Frame Standards relevant for our research.

\section{SWAN}
The core functionality of SWAN is to act as a middleware between the phone applications and the hardware or software sensors.
We will further refer to the expression which is being passed to SWAN from application as SWANSong expression, or simply SWAN Song.
There are multiple types of SWANSong Expressions: \label{swan_song_expressions}
\begin{itemize}
 \item Value Expression - Retrieve values from sensors
 \item Tristate Expression - Perform evaluation on the values before sending them to applications
\end{itemize}

The application's interface with SWAN is always the same, the only component that varies is the SWANSong expression passed to SWAN.
SWANSong Expression (\hyperref[fig:SwanExpression]{Figure 3.1}) encapsulates all the information required by SWAN.
The main components of a SWANSong expression are:
\begin{itemize}
 \item Location - Tell SWAN were to evaluate the expression
 \item Sensor - Name of the SWAN Sensor
 \item Value Path - Identifies which value is requested
 \item Configuration - Web and local storage options
 \item Evaluation Options - Parameters for Evaluation Engine
\end{itemize}

Besides passing data to the application, SWAN also takes care of evaluation and storage. Relevant parameters are also embedded into the 
SWANSong expression.


\begin{figure}[htbp]
  \centering
    \includegraphics[scale=0.6]{Figures/swan_expr.pdf}
    \rule{35em}{0.5pt}
  \caption[Swan Expression]{Detailed SWAN expression}
  \label{fig:SwanExpression}
\end{figure}

As part of our research we will also apply changes on the SWAN Song Expression and the meaning of different parameters.

\section{Beacon Frame Standards}
With Bluetooth Low Energy devices market in its incipient stage, the Beacon  emitters suffer from high fragmentation.
To avoid the fragmentation of the Beacon Market, Google and Apple stepped in and proposed two standards for Beacon Frame Layout:

\begin{itemize}
 \item Apple iBeacon - simple beacon format, mostly used for proximity(distance measuring) applications
 \item Google Eddystone - complex standard, with 3 available frame formats:
 \begin{itemize}
  \item Eddystone UID - similar to iBeacon, broadcast unique ID
  \item EddystoneTLM - telemetry frame format, stores information about the beacon, such as temperature, battery level, number of packets sent
  \item Eddystone URL - frame layout which encodes a 17 bytes long URL in the frame
 \end{itemize}
\end{itemize}

The standards from above are industry recognized and widely implemented by various Beacon Manufacturers.
Unfortunately, even the Google Eddystone Format is not flexible enough and some companies need to come up with their own format to support extra functionality 
added to  their beacon products. We will also discuss the following frame formats which are proprietary to companies selling the test beacons, but they enable us to explore the
new way of using beacons:
\begin{itemize}
 \item AltBeacon Beacon Format - default beacon format present in the library used by SWAN for beacon scanning
 \item Estimote Nearable - Beacon Frame Format developed by Estimote\cite{estimote_company}, with accelerometer and movement data embedded into the format
\end{itemize}

% Chapter Template

\chapter{SWAN Data Specifications}\label{ssec:swandataspecifications} % Main chapter title

\label{Chapter4} % Change X to a consecutive number; for referencing this chapter elsewhere, use \ref{ChapterX}

\lhead{Chapter 4. \emph{SWAN Data Specifications}} % Change X to a consecutive number; this is for the header on each page - perhaps a shortened title

%----------------------------------------------------------------------------------------
%	SECTION 1
%----------------------------------------------------------------------------------------

\section{Introduction}
Initially SWAN was designed to take a single value from the sensor data.
The value should be a primitive type such as Integer, Float, Double etc.
This constraint applies because SWAN evaluation engine allows the programmer to perform MIN, MAX and AVERAGE operations on given sensor data.
Also, History reduction is important when using the single value.

By adding multiple sensors, it became clear that taking a single value is not scalable enough. 
The further changes were delayed by adding multiple value paths to the sensor implementation. 
A clear example was accelerometer data: You will probably need the values for all 3 axes: x, y, z.
And you need to register 3 expressions for each value path.
The limitation can no longer be avoided after implementing Beacon Sensors, because now we need to add a full array of beacon identifiers under the same timestamp,
the key-value to provide individual temperature data or even key-array of values if we want the accelerometer data from all beacons.

%-----------------------------------
%	SUBSECTION 1
%-----------------------------------
\section{Proposed Solutions}
All the proposed solutions should meet the following requirements:
\begin{itemize}
 \item Support multiple sensors of the same type (beacon sensor)
 \item Support multiple values for each sensor
 \item Provide scheme for single sensor single value data
 \item Provide implementation for Evaluation Engine Application
\end{itemize}

\subsection{Data Mapping approach}
Data Mapping tries to solve the problem by proposing a new, novel approach of how data is being passed form the sensor
to the evaluation engine. The model is aiming to solve future problems when it comes to supporting multiple sensors
with multiple values at the same time. The drawback of this solution is the number of changes that are required to be done in
Evaluation Engine and sensors, in order to implement the new approach. Also there is a high risk of breaking compatibility
with older expressions.

The format of data for the following function calls(value parameter):
\begin{lstlisting}[language=Java]
 AbstractSwanSensor.putValueTrimSize(final String valuePath,
					final String id,
					final long now,
					final Object value):
\end{lstlisting}

In order to satisfy the new requirements, we propose the new data format to be encoded as: \begin{verbatim} Map<ValuePath, Map<sensorID,  ArrayList<Object>> \end{verbatim}

where:
\begin{itemize}
 \item ValuePath  = sensor valuepaths, ex Acceleromter x, y, z, total
 \item sensorID  =  self or unique Sensor Identifier(BeaconID)
\end{itemize}

We identify four sensor formats and recommend the following representation for each type of sensor:
\begin{itemize}
  \item Single Sensor Single Value(sensor name = stepcounter):
  \begin{itemize}
    \item \begin{verbatim} Example: Map: { steps={self=[13434545 ] }} \end{verbatim}                                                              
  \end{itemize}
 
  \item  Single Sensor, Multiple Values(sensor name =accelerometer):
  \begin{itemize}
    \item Key - name of the sensor
    \item Array of values - multiple values in the same order
    \item \begin{verbatim} Example: Map: { x={self=[0.5]},
                 y={self=[0.6]}, z={self=[-0.5]} } \end{verbatim}  
  \end{itemize}

 
  \item Multiple Sensor, Single Value(Beacon Distance):
  \begin{itemize}
    \item Key - the sensor identifier(id)
    \item Array of values - single value in the array
    \item  \begin{verbatim}  Example: Map: {distance={beaconID1=[1.5],
                 beaconID2=[0.5], beaconID3=[0.2]}} \end{verbatim}  
  \end{itemize}

  \item  Multiple Sensor, Multiple Values(sensor name = Beacon Accelerometer):
  \begin{itemize}
    \item Key - the sensor identifier(id), should be unique
    \item Array of values - multiple values in the same order
    \item  \begin{verbatim}  Example: Map:{x ={beaconID1=[-0.5], beaconID2=[-0.1],
                y = {beaconID1=[ 0.2], beaconID2=[-0.4]}} \end{verbatim}  
 \end{itemize}


\end{itemize}




%-----------------------------------
%	SUBSECTION 2
%-----------------------------------

\subsection{Expression Location Approach} \label{sssec:exprlocation}
Compared to the Data Mapping, Expression Location Approach tries to preserve the single return value and value paths for each sensor.
It works in addition to the current implementation and maintain the backwards compatibility with the previous expression format.
The drawback of this approach is very low flexibility when the number of sensors are really high. Also , it tries to solve an immediate 
problem rather than focusing on future challenges.

Instead of using a map to send the values to the evaluation engine, we add an extension to the current location of the swan expression.
The evaluation engine recognize keywords such as \textbf{self}, \textbf{wear} and \textbf{remote}. To support the multiple sensors, 
we add the sensor's unique identifier instead of the location. In case of Beacons, each beacon has beacon Id, which should be unique for 
all kind of applications.

The recommended swan expressions for the 4 categories of sensors:

\begin{itemize}
  \item Single Sensor Single Value(sensor name = stepcounter):
  \begin{itemize}
    \item Register an expression: \begin{verbatim}  self@stepcounter:steps{ANY, 1000}\end{verbatim} 
    \item Use putTrimValue() with a single value
  \end{itemize}
 
  \item  Single Sensor, Multiple Values(sensor name =accelerometer):
  \begin{itemize}
    \item Register an expression with value path: \begin{verbatim} self@accelerometer:x{ANY, 1000}\end{verbatim}
    \item Use putTrimValue() with a single value
  \end{itemize}

  \item Multiple Sensor, Single Value(Beacon Distance):
  \begin{itemize}
    \item Use Beacon Discovery Sensor to get a list beaconIDs: \begin{verbatim} self@beacon_discovery:ibeaconuuid{ANY, 10}\end{verbatim} 
  \end{itemize}

  \item  Multiple Sensor, Multiple Values(sensor name = Beacon Accelerometer):
  \begin{itemize}
    \item Use Beacon Discovery Sensor to get a list beaconIDs: \begin{verbatim} self@beacon_discovery:ibeaconuuid{ANY, 10}\end{verbatim}
    \item Register individual expressions with individual valuepaths for each BeaconID: 
    \begin{itemize}
     \item \begin{verbatim} beaconID@beacon_movement:x{ANY,1000} \end{verbatim}
     \item \begin{verbatim} beaconID@beacon_movement:y{ANY,1000} \end{verbatim}
    \end{itemize}

 \end{itemize}


\end{itemize}

Note these expressions can not be evaluated by Evaluation Engine, because it outputs an array instead of a single value.

%----------------------------------------------------------------------------------------
%	SECTION 2
%----------------------------------------------------------------------------------------

\section{Implemented Approach}
The implemented solution was the \hyperref[sssec:exprlocation]{Expression Location Approach}.
The while the Data Mapping approach brings more novelty, it comes with severe disadvantages:
\begin{itemize}
 \item Unable to keep backward compatibility with older applications which are using SWAN - before the changes, applications
 expect a single primitive value. Using an array or a map may result in application crashes.
 \item While the Value Based Expressions are easy to adapt, changing Tristate Expressions is more difficult - Evaluation engine needs to be redesigned to adapt the new approach
 \item Big number of changes may render SWAN unstable and it will take a long time to narrow down bugs
\end{itemize}

On the other hand, the Location Based does not interfere with Evaluation Engine's algorithm, and location can be easily passed to the sensor implementation and 
handled by sensor implementations. 

Implementing the location extension was easy, we just added an extra parameter of type \textbf{Map} which we pass to the sensor
when we bind it.
The existing sensors preserved their implementation without changes and the beacon based sensors were adapted to get the beacon ID from the location parameters.
Also, for Bluetooth sensor discovery purpose, we left the Discovery sensor to output an array instead of a single value, assuming that no Tri-State Expression will be registered with that 
sensor.
 
% Chapter Template

\chapter{Implementation Details} % Main chapter title

\label{Chapter5} % Change X to a consecutive number; for referencing this chapter elsewhere, use \ref{ChapterX}

\lhead{Chapter X. \emph{Chapter Title Here}} % Change X to a consecutive number; this is for the header on each page - perhaps a shortened title

%----------------------------------------------------------------------------------------
%	SECTION 1
%----------------------------------------------------------------------------------------

\section{Swan Application Layout}

\section{Beacon based Bluetooth Sensors}

\section{Value based SmartWatch Sensors}

\section{SWAN based Sensors for SmartWatch}

 
% Chapter Template

\chapter{Power Efficiency Analysis} % Main chapter title

\label{Chapter6} % Change X to a consecutive number; for referencing this chapter elsewhere, use \ref{ChapterX}

\lhead{Chapter 6. \emph{Power Efficiency Analysis}} % Change X to a consecutive number; this is for the header on each page - perhaps a shortened title

%----------------------------------------------------------------------------------------
%	SECTION 1
%----------------------------------------------------------------------------------------

\section{Introduction}
The current implementation of the SWAN WEAR allows us to choose which way we want to process the data from the smartwatch application:
\begin{itemize}
 \item Perform minimum amount of computation on the watch and just send values to be processed by the SWAN PHONE application
 \item Send the expression to SWAN WEAR application for evaluation
\end{itemize}

Typically, the recommended approach is to minimize the computation on the watch to save battery, but in certain cases,  when expression does not require frequent evaluation, we might get better power savings if we choose to perform the evaluation on the watch.

\section{Design}
Before proceeding to the data analysis, according to the GQM\cite{gqm_1}\cite{gqm_2} paradigm, we should define the goal, the questions and the metrics.
    Our goal is to \textbf{analyse the power consumption} of phone and smartwatch, for the purpose of \textbf{comparing two different methods of acquiring sensor data}
    with respect to \textbf{differences in power consumption of phone and smartwatch}, in context of \textbf{SWAN android application}.

\textbf{Question 1}: What is the overhead of gaining data from smartwatch, compared to running evaluation on the watch?
\begin{itemize}
  \item Metric 1:  Battery level on Android Phone
  \item Metric 2:  Battery level on Android Smartwatch
  \item Metric 2:  Battery level on Android Smartwatch
  \item  Metric 3:  Expression runtime in seconds
\end{itemize}

\textbf{Question 2}:  What is the overhead of gaining data from test sensor compared to real hardware sensor?
\begin{itemize}
 \item Metric 1:  Battery level on Android Phone
 \item Metric 2: Battery level on Android Smartwatch
\end{itemize}

\section{Experiment Planning}
\subsection{Context Selection}
The experiment will be run on the simulated environment, composed of Android Phone and Android Wear Smartwatch. We will consider our experiment a real life problem because:
\begin{itemize}
 \item The tests are performed on real devices, which are used as reference for many android developers
 \item The test suite is composed of various sensor data, which reflect the actual SWAN performance in real life applications.
\end{itemize}

\subsection{Variable Selection}

The main dependent variable is \textbf{power consumption} expressed as battery power consumed after the experiment.
The independent variables are \textbf{data acquisition method} and \textbf{delay}.
Data acquisition method  have two options:
\begin{itemize}
 \item Phone based SWAN expressions
 \item Wear Based SWAN Expressions
\end{itemize}

Delay is set by us, but to avoid spending too much time on experiment, we will choose between two options: fast(100ms) and slow(1 second).

\subsection{ Hypothesis Formulation}

    We consider $\mu$ the battery power consumed by the test suite. Since we have two devices to measure power consumption from,
    the $\mu$ will be calculated as $\delta$\textsubscript{phone} +  $\delta$\textsubscript{wear}, where $\delta$ is the difference in battery levels 
    before and after the experiment.
    
    Special case:\label{special_case} If the power consumption is better on phone but worse on watch for the given approach, we cannot test the hypothesis:
    \begin{itemize}
     \item ($\delta$\textsubscript{wear test type 1} -  $\delta$\textsubscript{wear test type 2}) \textgreater  0
     \item  ($\delta$\textsubscript{phone test type 1} -  $\delta$\textsubscript{phone test type 2}) \textless  0 
    \end{itemize}
    
    The reason why we can't apply our hypotheses is because the smartwatch and the phone have different hardware, idle power consumption and 
    battery installed. We cannot compare them directly, but we can claim that one approach is better for phone but worse for watch and let the developers
    make decision on what they want to value the most: Phone or watch battery.

    \textbf{Question 1:} What is the overhead of gathering data from smartwatch, compared to running evaluation on the watch? \newline
Conjecture(P): There is a difference between gaining data and evaluating expression on the smartwatch is significant \newline
Consequence(Q): The difference between only gaining data and evaluating expression on the smartwatch. \newline
\textit{Null hypothesis} - No observable difference in power consumption between only gathering data and evaluating expression on the watch \newline
H0: ( $\mu$\textsubscript{phone based} - $\mu$\textsubscript{wear based} = 0) \newline
  Alternative hypothesis - There is a difference in power consumption between only gaining data on watch and running a SWAN expression on the watch \newline
H1:($\mu$\textsubscript{phone based} - $\mu$\textsubscript{wear based} $\neq$ 0)  \newline

    \textbf{Question 2:}  What is the overhead of gaining data from test sensor compared to real hardware sensor?\newline
    Conjecture(P): There is  a difference between using test sensor and real sensor on the smartwatch is significant\newline
    Consequence(Q): The difference between only gaining data and evaluating expression on the smartwatch is significant.\newline
    \textit{Null hypothesis} - No observable difference in power consumption between using test sensor and hardware sensor on the watch\newline
H0: ($\mu$\textsubscript{test} - $\mu$\textsubscript{real} = 0)\newline
    Alternative hypothesis - There is a difference in power consumption between using test sensor and hardware sensor on the watch\newline
H1:($\mu$\textsubscript{test} - $\mu$\textsubscript{real} $\neq$ 0)\newline

\subsection{Subject Selection}
For this experiment we will use a batch of expressions targeting multiple sensors available on the smartwatch.
The batch for phone based tests contains the following expressions:
\begin{itemize}
 \item \begin{verbatim}self@wear_movement:x?delay={$delay}$server_storage=FALSE{ANY,0}\end{verbatim}
 \item \begin{verbatim}self@wear_gamerotation:x?delay={$delay}$server_storage=FALSE{ANY,0}\end{verbatim}
 \item \begin{verbatim}self@wear_linearacceleration:x?delay={$delay}$server_storage=FALSE{ANY,0}\end{verbatim}
 \item \begin{verbatim}self@wear_gravity:x?delay={$delay}$server_storage=FALSE{ANY,0}\end{verbatim}
 \item \begin{verbatim}self@wear_heartrate:heart_rate?delay={$delay}$server_storage=FALSE{ANY,0}\end{verbatim}
\end{itemize}

The batch for wear based expressions contains the following expressions:
\begin{itemize}
 \item \begin{verbatim}wear@movement:x?delay={$delay}$server_storage=FALSE{ANY,0}\end{verbatim}
 \item \begin{verbatim}wear@gamerotation:x?delay={$delay}$server_storage=FALSE{ANY,0}\end{verbatim}
 \item \begin{verbatim}wear@linearacceleration:x?delay={$delay}$server_storage=FALSE{ANY,0}\end{verbatim}
 \item \begin{verbatim}wear@gravity:x?delay={$delay}$server_storage=FALSE{ANY,0}\end{verbatim}
 \item \begin{verbatim}wear@heartrate:heart_rate?delay={$delay}$server_storage=FALSE{ANY,0}\end{verbatim}
\end{itemize}

The \textbf{\{\$delay\}} will be replaced by appropriate value when registering an expression

To evaluate the performance of the real sensor compared to test sensor, we will run the test sensor for the same 
amount of values and we will compare its power consumption with power consumption from hardware sensors. 
There will be 2 implementations of the test sensor, one for each data acquisition approach.

\subsection{Experiment Design}
For the first part of the experiment we will have two factors to vary: \textbf{data acquisition approach} and \textbf{delay}.
This will give us four different scenarios to test:

\begin{center}
  \begin{tabular}{ |l|l|l|l| }
  \hline
  \multicolumn{4}{ |c| }{Factor A: Data acquisition Method} \\
  \hline
  \multicolumn{2}{|c|}{Phone Based Evaluation}  & \multicolumn{2} {|c|}{Wear Based Evaluation} \\
  \hline
  \multicolumn{2}{|c|}{Factor B: Delay}  & \multicolumn{2} {|c|}{Factor B: Delay} \\
  \hline
  Delay: 100ms & Delay: 1000 ms & Delay: 100ms & Delay: 1000 ms\\
  \hline
  3x Experiments & 3x Experiments & 3x Experiments &3x Experiments\\
  \hline
  \end{tabular}
\end{center}

The second part of the experiment will focus on quantifying the discrepancy between power consumption when using a test sensor
versus using a real, hardware sensor. In this test scenario we also apply delay of 100 ms and 1000ms, to be fully aware of the implications of all factors.

\begin{center}
 \textbf{Delay 100 ms:}
\end{center}

\begin{center}
  \begin{tabular}{ |l|l|l|l| }
  \hline
  \multicolumn{4}{ |c| }{Factor A: Data acquisition Method} \\
  \hline
  \multicolumn{2}{|c|}{Phone Based Evaluation}  & \multicolumn{2} {|c|}{Wear Based Evaluation} \\
  \hline
  \multicolumn{2}{|c|}{Factor B: Sensor Type}  & \multicolumn{2} {|c|}{Factor B: Sensor Type} \\
  \hline
  Real Sensor & Test Sensor & Real Sensor & Test Sensor\\
  \hline
  3x Experiments & 3x Experiments & 3x Experiments &3x Experiments\\
  \hline
  \end{tabular}
\end{center}

\begin{center}
 \textbf{Delay 1000 ms:}
\end{center}

\begin{center}
  \begin{tabular}{ |l|l|l|l| }
  \hline
  \multicolumn{4}{ |c| }{Factor A: Data acquisition Method} \\
  \hline
  \multicolumn{2}{|c|}{Phone Based Evaluation}  & \multicolumn{2} {|c|}{Wear Based Evaluation} \\
  \hline
  \multicolumn{2}{|c|}{Factor B: Sensor Type}  & \multicolumn{2} {|c|}{Factor B: Sensor Type} \\
  \hline
  Real Sensor & Test Sensor & Real Sensor & Test Sensor\\
  \hline
  3x Experiments & 3x Experiments & 3x Experiments &3x Experiments\\
  \hline
  \end{tabular}
\end{center}

\subsection{Threats to Validity}
External: The experiment is being performed on specific setup. Even if the devices in our tests are chosen to represent the reference,
the android market is fragmented, and different android versions, hardware and smartwatches can yield to a different result

Internal: Basing our power consumption  results on battery levels increase the total error rate. Nonlinear discharge rate,
battery wear level\cite{battery_wear_proc} can influence the final conclusion. To avoid these battery limitations, we will perform tests only when the battery is fully charged, 
so the discharge pattern will be the same. Also to reduce the battery wear impact, the tests will be performed with limited time delay,
and without any phone or smartwatch usage between the tests.

 Internal: The batch execution of the expression can induce extra error if different types of sensors have different power consumption. 
 We will try to limit the impact of this error type by proving that the selection of the sensors has the same power draw by
 performing single sensor benchmark with initial full battery. 
 
 Internal: The communication using Bluetooth may vary energy consumption based on distance or obstacles between devices.
 We minimize the impact of Bluetooth, by placing phone and smartwatch in close proximity with no obstacles in between.
 
 External: Heart rate sensor can be unpredictable. In some circumstances, the heart rate sensor can stop giving values. 
 Since we were not wearing the watching during our test, were placed it on the charging dock, with no charging cable connected and the phone
 in close proximity. By satisfying this conditions, the experiment setup and results can be verified.
 
 \subsection{Instrumentation}
 The following devices are available to perform our experiment:
 \begin{itemize}
  \item Phone - Nexus 6P - manufactured by Huawei,  with latest Android 6.0 Marshmallow installed and with Android security patch level: 1 June 2016. 
  \item Smartwatch - Motorola 360 gen 2, with latest Android 6.0 Marshmallow, android wear version 1.5 
 \end{itemize}

 We could have a bigger selection of phones to run SWAN on them, since we only have one smartwatch available,
 the different processors  and Bluetooth chips might induce more random in our observations.
 Besides we can argue that Nexus phones are used by default by lots of smartwatch manufacturers for tests.
 Before proceeding to experiment setup, the devices were factory reset and the only android wear app necessary for smartwatch connection was installed.
 We observed that by default a lot of Google applications are installed on the Nexus Phone. Some of them also have watch apps packaged.
 To avoid possible battery drain from other applications we decided to disable the following apps on the Nexus Phone:
 \begin{itemize}
  \item  Google Drive
  \item Google Play Games
  \item Google Play Music
  \item Hangouts
  \item Google Maps
  \item Google Photos
  \item Youtube
  \item Google Docs
 \end{itemize}

 Measuring power consumption of both approaches require us to have accurate tools for measuring power consumption.
 The power consumption should be measured on both phone and android wear device, so we could better understand what are the pros and cons of using each method.
 We have two methods of measuring the power consumption:
 \begin{itemize}
  \item Using the android battery starts, which provide the current power draw if the device is equipped with a hardware fuel gauge(Our test devices Nexus 6P and Moto 360 gen 2 )\cite{fuel_gauge}
  \item The standard setup: Running each method for a prolonged amount of time and measure the battery level after the experiment is done
 \end{itemize}

 Using the hardware power gauge is a very useful feature to measure the power consumption. Unfortunately, it suffers from two major disadvantages:
 \begin{itemize}
  \item We require to run another battery sensor, so the results may not reflect the constant power consumption of the swan application running
  \item Hardware fuel gauge has its own limitations, and the computed power usage may not be accurate
 \end{itemize}

 Using battery level measurement can also be affected by battery maximum capacity, charge-discharge cycle, but since we repeat the experiment for both options on the same hardware, and without delay we can argue that results can be compared.
 Additionally, measuring the battery levels allow us to have a better average value.
 
 \subsection{Experiment Execution}
 After initial research we concluded that the data from hardware fuel gauge is highly inaccurate on Android Wear.
 The reason is exceptionally power efficient hardware and high error rate for power consumption sensor. 
 Also we are unable to test what is the power implication of running the SWAN WEAR foreground service. 
 In order to obtain some power metrics, we will monitor battery levels on phone and smartwatch after SWAN expression returns a given amount of sensor values, in both scenarios.
The results that we might will be analysed based on power consumption on both phone and smartwatch.
The experiment steps should be performed in the following order:
\begin{itemize}
 \item Factory reset both phone and smartwatch.
 \item Pair smartwatch to the phone
 \item Install swan on the phone and smartwatch
 \item Run a preliminary test, to see if swan is ready to operate
 \item Charge both phone and smartwatch to 100%
 \item Install test application
 \item Disable Wi-Fi and radio, leave only Bluetooth
 \item Start test application and turn off the screen of the phone
 \item Test application should take the phone and wear battery levels and remaining Power values before starting test evaluations
 \item Wait for application to finish the test suite
 \item Test application will take the battery levels and remaining power values after the tests are finished and will write them on local storage for further analysis
\end{itemize}

For the testing purpose, we will avoid sensors that are dependent of user motion to actually work: 
\begin{itemize}
 \item Step counter sensor - will send updates only if the user walked after last query. It might also be related to hardware implementation of the sensor in our test device\cite{motorolla_stepcounter}. 
 \item Light sensor - the updates are triggered by changes in light level and since our test setup is configured to run for a long time, we cannot include a sensor triggered by external,
 undefined behaviour
\end{itemize}

Surprisingly, the Heart Rate Sensor will still give you values even if it is not mounted on the wrist, so we can use it for our test purpose. 
Heart rate sensor is different compared to regular sensors found on the phone. The smallest granularity for heart rate results is one second,
so specifying a smaller delay will not make any sense.  Delays bigger than one second instead will fulfilled by SWAN implementation.
 
 \subsection{Preliminary Tests}
Preliminary tests shows us that we cannot rely on the estimated power reported by the fuel gauge sensors. In case of our phone device,
Nexus 6P, the reported values in Nano-watts were not accurate at all. Instead, the current discharge current was reported, 
but we can not use it in our tests, since it will interfere with our test setup.
We encountered the similar problem on our smartwatch, Motorola 360 generation 2. 
The reported value in Nano-watts was not corresponding with the maximum theoretical energy that the watch battery is able to hold, according to the specifications.
The current power consumption value.
On the other hand, the formulas for running the specified amount of time are accurate, we recorded a deviation of +- 5\% of run time for different types of sensors 

To get comparable results we are supposed to guarantee equal conditions for both test scenarios.
Since SWAN is quite power efficient and sensors that we are targeting are also very power efficient, the test should run for a long period of time.
We estimated that a notable difference in battery levels should occur after running SWAN expressions for around two hours' period.
Running tests for longer period of time can help us identify even smaller differences, 
but the necessity of running the experiments with different type of expressions and also the number of repetitions for each scenario make the use of longer runtimes prohibitive.
Also we should take into consideration the charging time necessary after each run of the experiment,
with smartwatch normally losing the biggest battery percentage  and having long recharge times.

\subsection{Test Application}
The test application was built in order to run tests in batch order and gather the battery level data. To avoid power consumption from external sources, such as phone screen, we implemented the testing code as foreground service\cite{foreground_service} which is not hibernated by the android system when the screen is off. 
The testing application has the following workflow:
For each SWAN expression:
\begin{itemize}
 \item Register a swan expression and get the battery level of the phone before the experiment
 \item Register a swan expression and get the battery level of the smartwatch before the experiment
 \item Register the start time
 \item Calculate the number of values necessary for the run, based on the given expected runtime in microseconds
 \item Register the experiment expressions and wait until the number of values are reached
 \item Register expression end time
 \item Get the battery level of the phone after the experiment
 \item Get the battery level of the smartwatch after the experiment
\end{itemize}


Once the tests for expression is done, the application will write on local storage, in a text file the expressions name,
battery levels before and after for each device the number of values received and the total runtime.
 The values are comma separated, so it can be imported in Excel for further processing.

Even if the program runs in a separate service, we are not allowed to block in the main thread. 
We are actually required to wait for the SWAN expression to terminate, so we sent the computation in another thread and we used Latch to wait for expression execution.

 \section{Results}

We performed all experiments according to the Experiment Design and we obtained a lot of data. To improve readability,
we decided not to include all the raw data in this chapter. If you are interested to see the raw data, check Appendix A, for tables containing the
average values of 3 experiments.

\subsection{Preliminary Analysis}

We present the results of our data in graphs below.
The Figures 6.1-6.5 contain the data relevant for our hypothesis testing.
 
 \begin{figure}[htbp]
  \centering
    \includegraphics[scale=0.8]{Figures/phone_vs_wear_100.pdf}
    \rule{35em}{0.5pt}
  \caption[Phone vs Wear Based Expressions, delay 100 ms]{Phone vs Wear Based Expressions, delay 100 ms}
  \label{fig:phone_vs_wear_100}
\end{figure}

\hyperref[fig:phone_vs_wear_100]{Figure 6.1} represents the power consumption of phone based expressions and wear based expression for very fast delay,
and fixed number of received values. As we can see the phone based SWAN expressions are highly inefficient, phone battery dropped \textbf{34\%} more and watch battery dropped 
\textbf{20 \% } more than watch based SWAN expression. 

 \begin{figure}[htbp]
  \centering
    \includegraphics[scale=0.8]{Figures/phone_vs_wear_1000.pdf}
    \rule{35em}{0.5pt}
  \caption[Phone vs Wear Based  Expressions, delay 1000 ms]{Phone vs Wear Based  Expressions, delay 1000 ms}
  \label{fig:phone_vs_wear_1000}
\end{figure}

\hyperref[fig:phone_vs_wear_1000]{Figure 6.2} represents the power consumption of phone based expressions and wear based expression for normal delay, also
with fixed number of received values. This test scenario highlights the very good power savings for phone battery, with only \textbf{1\%} drop in phone battery for watch based SWAN
expressions. Phone based expressions perform better than in fast delay scenario, but still very power inefficient for both watch and phone.

 \begin{figure}[htbp]
  \centering
    \includegraphics[scale=0.8]{Figures/execution_times.pdf}
    \rule{35em}{0.5pt}
  \caption[Run time of different number of sensors, baseline 4800 seconds]{Run time of different number of sensors, baseline 4800 seconds}
  \label{fig:execution_times}
\end{figure}

As part of the experiment, we noticed that different types of expression have different runtimes for the same number of recorded values. The expected run time for all expression,
given the delay, was 4800 seconds. \hyperref[fig:execution_times]{Figure 6.3} shows the execution times of wear and phone based expressions, for both real sensors and our software test sensor.
We might argue that longer runtimes translate into more battery consumption, but we consider that it is important to honour the given delay and extra power consumption will still be counted for 
our analysis.

 \begin{figure}[htbp]
  \centering
    \includegraphics[scale=0.8]{Figures/phone_expr_consumption.pdf}
    \rule{35em}{0.5pt}
  \caption[Expression Power Consumption]{Phone Based Expression Power Consumption}
  \label{fig:phone_expr_consumption}
\end{figure}

The graphs show clearly that phone based expressions are not efficient.  We should further validate our approach by excluding the variable update rate from a real hardware sensor.
Test sensor will always send updates with the specified delay, and as we can see in \hyperref[fig:phone_expr_consumption]{Figure 6.4}, the shorter delay from hardware sensor is responsible
for more power used.

 \begin{figure}[htbp]
  \centering
    \includegraphics[scale=0.8]{Figures/wear_expr_consumption.pdf}
    \rule{35em}{0.5pt}
  \caption[Expression Power Consumption]{Phone Based Expression Power Consumption}
  \label{fig:wear_expr_consumption}
\end{figure}

For watch based expression, the phone battery improvements are non-existent(or, at least do not surpass our error threshold the be considered a difference). On the other hand,
a stable delay still translates into \textbf{ 5 to 10\%} less battery consumed on the watch as we can see in \hyperref[fig:wear_expr_consumption]{Figure 6.5}

\section{Hypothesis Testing}
After preliminary data analysis we notice that we don't have the \hyperref[special_case]{special case}, so we can proceed to test out hypothesis.

\textit{Research Question 1:} What is the overhead of gathering data from smartwatch, compared to running evaluation on the watch? \newline
Following the formulas described in Hypothesis Formulation section, we observe that for delay of 100 ms, phone based expressions consume 
\textbf{34\% more} phone battery and \textbf{~20\% more } smartwatch battery. Taking into account the 5\% error by the imperfection of our instruments,
we reject the Hypothesis H0, and accept the Hypothesis H1, for the given delay of 100 ms.

For given delay of 1000ms, we observe that for delay of 100 ms, phone based expressions consume 
\textbf{17\% more} phone battery and \textbf{~20\% more } smartwatch battery. Taking into account the 5\% error by the imperfection of our instruments,
we reject the Hypothesis H0, and accept the Hypothesis H1.

 \textit{Research Question 2:}  What is the overhead of gaining data from test sensor compared to real hardware sensor?\newline
Following our further research of why the phone based expressions consume so much power, we wanted to test the sensor update delay, as a factor
which increase the power consumption. Our test sensor implementation always honour the delay specified in the expression, compared to a very weak delay guarantee
given by a real hardware sensor.

For given delay of 100ms, hardware sensor consumed \textbf{14\% more} phone battery and \textbf{~17\% more } smartwatch battery than the test sensor, assuming that both test and hardware
sensor is implemented as phone based expression.
Taking into account the 5\% error by the imperfection of our instruments, we reject the Hypothesis H0 and accept the Hypothesis H1.

For given delay of 1000ms, hardware sensor consumed \textbf{15\% more} phone battery and \textbf{~22\% more } smartwatch battery than the test sensor, assuming that both test and hardware
sensor is implemented as phone based expression.
Taking into account the 5\% error by the imperfection of our instruments, we reject the Hypothesis H0 and accept the Hypothesis H1.

Comparing the wear based expressions is more complicated. We could not record any notable differences on phone battery during our tests, so we will focus on smartwatch battery.

For given delay of 100ms,  hardware based sensor consumed \textbf{3\% more} watch battery than the test sensor.
Our 5\% error rate require to have a difference of at least 1.5\% in battery levels (5\% of 30\% battery drop is equal to 1.5\%). We
reject hypothesis H0 and accept Hypothesis H1.

For given delay of 1000ms,  hardware based sensor consumed \textbf{9\% more} watch battery than the test sensor.
Our 5\% error rate require to have a difference of at least 1\% in battery levels (5\% of 20\% battery drop is equal to 1\%). We
reject hypothesis H0 and accept Hypothesis H1.

\subsection{Final Results Discussion}

After performing all the tests in out suite, we revealed that using phone based expression in current SWAN implementation is not a good idea.
Specifically, the phone is getting hot on low delays and the phone battery drop is one order of magnitude higher than using SWAN expression evaluation on the watch.

After careful analysis we identified two possible reasons why phone based expression consumes so much power, but also what can further improve the smartwatch based expressions:
\begin{itemize}
 \item The hardware sensor does not honour the requested delay
 \begin{itemize}
  \item A lot of values are being sent over Bluetooth and the phone has to drop them because they do not arrive in the given timestamp
  \item Wear based expression drop values locally, so no extra values are sent over Bluetooth
  \item Phone based test sensor show much better battery when the values arrive at given delay
 \end{itemize}
 \item SWAN mechanism of accepting values drops too many values, that were supposed to be accepted
 \begin{itemize}
  \item The Bluetooth and synchronization delay change the arrival time of the values, and the value accept function takes into consideration arrival time, not the time when values were recorded
  \item Test Sensor revealed poor performance of the accept value implementation, which translates two times longer runtimes than predicted in our test scenarios.
  \item Having the expected runtime as close as possible to real runtime can save battery
 \end{itemize}

 The accept values implementation performance needs to be further investigated.
 As the result of our experiments we can conclude that the test sensor, with ideal delay, have lower battery consumption.
 The excessive runtime as side effect of the test sensors, and log analysis(we record when the values are being dropped) give us some clues to the problem, but we didn't manage to fully isolate the
 problem so we could say for sure that the accept method is responsible for long runtimes.

\end{itemize}
 
% Chapter Template

\chapter{Conclusion and Future Work} % Main chapter title

\label{Chapter7} % Change X to a consecutive number; for referencing this chapter elsewhere, use \ref{ChapterX}

\lhead{Chapter 7. \emph{Conclusion and Future Work}} % Change X to a consecutive number; this is for the header on each page - perhaps a shortened title

%----------------------------------------------------------------------------------------
%	SECTION 1
%----------------------------------------------------------------------------------------

\section{Conclusions}
The master thesis presents the use of new technologies, such as smart watches and beacons in context of mobile phone applications. We proposed and implemented two methods 
of accessing smartwatch features and provided an extensive power analysis of using them in real life scenarios. As results of our research we demonstrated that SWAN is able to run
on any Android Wear device. SWAN running on the smartwatch was able to surpass simple data collection in any test scenario in power efficiency,
providing superior power savings on both phone and watch.

The added support for beacon based sensors allowed us to provide specifications for data formats which can be used as guidelines for adding support of other sensor types. We proposed two
specification drafts and we opted for implementing the option which have the best compatibility with the existing implementation.

We noticed that beacon market suffers from very high fragmentation despite the effort from Google and Apple to offer a standard. However, we managed to overcome this obstacle by using
a single, open-source library for parsing. Moreover, we proved that even a closed source format such as Estimote Nearable can be easily analysed and supported without the use of the proprietary development kit.

%-----------------------------------
%	SUBSECTION 1
%-----------------------------------
\section{Future Work}
Our power saving tests revealed inconsistent run times for the same amount of requested values and constant delay. For some test cases we noticed a run time value twice as high as expected one.
Long run times can also translate in poor power efficiency. We were able to identify few problems with SWAN delay enforcing implementation, but the problem should be further investigated.

We were able to prove that multiple sensors of the same type can be integrated with SWAN, by integrating support for multiple beacons and implementing few test sensors to retrieve values.
Unfortunately, we were only able to validate the sensors with Swan Monitor\cite{swan_monitor} application. Building a real proof-of-concept application that use beacons to present real use cases should be done in near future.

 

%----------------------------------------------------------------------------------------
%	THESIS CONTENT - APPENDICES
%----------------------------------------------------------------------------------------

\addtocontents{toc}{\vspace{2em}} % Add a gap in the Contents, for aesthetics

\appendix % Cue to tell LaTeX that the following 'chapters' are Appendices

% Include the appendices of the thesis as separate files from the Appendices folder
% Uncomment the lines as you write the Appendices

% Appendix A

\chapter{Power Analysis Data} % Main appendix title

\label{AppendixPower} % For referencing this appendix elsewhere, use \ref{AppendixA}

\lhead{Appendix A. \emph{Power Analysis Data}} % This is for the header on each page - perhaps a shortened title

We recorded a lot of data as part of the power consumption experiment. We do not include it in the main thesis body, but we offer the reader
the possibility of analyzing and reading raw data.

All the tables from below include the average of 3 experiments performed. The average function used it the AVERAGE macro from Google Docs.

\begin{center}

  \begin{tabular}{ |p{2.5cm}|p{1.5cm}|p{1.5cm}|p{1.5cm}|p{1.5cm}|p{1.5cm}|p{1.5cm}|}
  \hline
Sensor Name &Number of values &	Phone battery before &	Smart watch battery before &	Phone battery after	& Smart watch battery after &	Runtime (sec) \\
  \hline
movement 	& 	12000	& 100	& 100	&  94.67	 &  86.67 	& 1521.67 \\
gamerotation	 & 	12000	&	94.67	&	86.67	& 	87.67	&	 74	& 1517.67 \\
linear acceleration 	& 12000	&	87.67	&	74 	&	80.67	& 62	& 	1507 \\ 
gravity 	&	12000	&	80.67	&	62	     	&		73.67	& 	49.33	& 	1515 \\
heartrate & 	1200	&	73.67	&	9.33		&	73.33	& 	46.33	& 	1288 \\
  \hline
  \end{tabular}
  \captionof{table}{Phone Based Expression with delay 100 ms} \label{tab:title1} 
\end{center}

\begin{center}
  \begin{tabular}{ |p{2.5cm}|p{1.5cm}|p{1.5cm}|p{1.5cm}|p{1.5cm}|p{1.5cm}|p{1.5cm}|}
  \hline
Sensor Name &Number of values &	Phone battery before &	Smart watch battery before &	Phone battery after	& Smart watch battery after &	Runtime (sec) \\
  \hline
movement 	& 	1200	& 100	& 100	& 96.67	& 89.33	&1178.33 \\
gamerotation	 & 1200	&96.67	& 89.33	& 92	& 79.33	& 1178  \\
linear acceleration 	& 1200	& 92	& 79.33	& 86.67	& 69.33	& 1178  \\ 
gravity 	&	1200	& 86.67	& 69.33	& 81.67	& 59.67	& 1178.33	 \\
heartrate & 	1200	& 81.67	& 59.67	& 80.67	& 55.33	& 1987.33 \\
  \hline
  \end{tabular}
    \captionof{table}{Phone Based Expression with delay 1000 ms} \label{tab:title2} 
\end{center}


\begin{center}
  \begin{tabular}{ |p{2.5cm}|p{1.5cm}|p{1.5cm}|p{1.5cm}|p{1.5cm}|p{1.5cm}|p{1.5cm}|}
  \hline
Sensor Name &Number of values &	Phone battery before &	Smart watch battery before &	Phone battery after	& Smart watch battery after &	Runtime (sec) \\
  \hline
movement 	& 12000	& 100	& 100	& 100	& 92	& 1254	 \\
gamerotation	& 12000& 	100	& 92	& 98.33	& 85	& 1248.67  \\
linear acceleration 	&  12000 &	98.33	& 85	& 97.33 	& 77 & 	1208.67\\ 
gravity 	&		12000	& 97.33& 	77	& 96	& 70	& 1253.67 \\
heartrate & 	1200	& 96	& 70 & 	95.33	& 67	& 1279 \\
  \hline
  \end{tabular}
      \captionof{table}{Wear Based Expression with delay: 100 ms} \label{tab:title3} 
\end{center}


\begin{center}
  \begin{tabular}{ |p{2.5cm}|p{1.5cm}|p{1.5cm}|p{1.5cm}|p{1.5cm}|p{1.5cm}|p{1.5cm}|}
  \hline
Sensor Name &Number of values &	Phone battery before &	Smart watch battery before &	Phone battery after	& Smart watch battery after &	Runtime (sec) \\
  \hline
movement 	& 	1200	& 100	& 100	& 100	& 94.33 & 	1189 \\
gamerotation	 &  1200	& 100	& 94.33	& 100	& 89.67	& 1176.67\\
linear acceleration 	& 1200	& 100	& 89.67	& 100	& 84.67	& 1203.67 \\ 
gravity 	&		1200	& 100	& 84.67	& 99	& 79.67	& 1190 \\
heartrate & 	 1200	& 99	& 79.67	& 98	& 75.33	& 2201.33\\
  \hline
  \end{tabular}
        \captionof{table}{Wear Based Expression with delay: 1000 ms} \label{tab:title4} 
\end{center}

\begin{center}
  \begin{tabular}{ |p{3cm}|p{1.5cm}|p{1.5cm}|p{1.5cm}|p{1.5cm}|p{1.5cm}|p{1.5cm}|}
  \hline
Sensor Name &Number of values &	Phone battery before &	Smart watch battery before &	Phone battery after	& Smart watch battery after &	Runtime (sec) \\
  \hline
movement 	& 	12000 & 	100 & 	100	& 95	& 87	& 1506 \\
gamerotation	 & 12000 & 100	& 100	& 95	& 87	& 1515  \\
linear acceleration 	& 12000	& 100	& 100	& 95	& 87	& 1513 \\ 
gravity 	&		12000	& 100	& 100	& 95	& 87	& 1514 \\
test sensor & 	12000	& 100	& 100	& 97	& 90	& 1862 \\
  \hline
  \end{tabular}
          \captionof{table}{Individual Phone Based Expression  power consumption, delay 100 ms} \label{tab:title5} 
\end{center}


\begin{center}
  \begin{tabular}{ |p{3cm}|p{1.5cm}|p{1.5cm}|p{1.5cm}|p{1.5cm}|p{1.5cm}|p{1.5cm}|}
  \hline
Sensor Name &Number of values &	Phone battery before &	Smart watch battery before &	Phone battery after	& Smart watch battery after &	Runtime (sec) \\
  \hline
movement 	& 	12000	& 100	& 100	& 100	& 91	& 1281 \\
gamerotation	 & 12000	 & 100	& 100	& 100	& 91	& 1265 \\
linear acceleration 	&  12000 & 	100	& 100	& 100	& 91	& 1207\\ 
gravity 	&		12000	& 100	& 100	& 100	& 91	& 1214 \\
test sensor & 	 12000	& 100	& 100	& 100	& 92	& 1325\\
  \hline
  \end{tabular}
            \captionof{table}{Individual Wear Based Expression  power consumption, delay 100 ms} \label{tab:title6} 
\end{center}

\begin{center}
  \begin{tabular}{ |p{3cm}|p{1.5cm}|p{1.5cm}|p{1.5cm}|p{1.5cm}|p{1.5cm}|p{1.5cm}|}
  \hline
Sensor Location &Number of values &	Phone battery before &	Smart watch battery before &	Phone battery after	& Smart watch battery after &	Runtime (sec) \\
  \hline
phone	&  48000 &	100	& 100	& 86	& 66.33	& 7480.33 \\
wear	 & 48000	 &100	&100	&95.67	&73.33 &	5617 \\
  \hline
  \end{tabular}
   \captionof{table}{Phone and Wear based Test Sensor, delay 100 ms} \label{tab:title7} 
\end{center}


\begin{center}
  \begin{tabular}{ |p{3cm}|p{1.5cm}|p{1.5cm}|p{1.5cm}|p{1.5cm}|p{1.5cm}|p{1.5cm}|}
  \hline
Sensor Location &Number of values &	Phone battery before &	Smart watch battery before &	Phone battery after	& Smart watch battery after &	Runtime (sec) \\
  \hline
phone	&  4800	& 100	&100	&97	 &82	&7663.67 \\
wear	 & 4800	&100	&100	&99	 &88.67	&4899 \\
  \hline
  \end{tabular}
   \captionof{table}{Phone and Wear based Test Sensor, delay 1000 ms} \label{tab:title8} 
\end{center}
% \begin{center}
%   \begin{tabular}{ |p{3cm}|p{1.5cm}|p{1.5cm}|p{1.5cm}|p{1.5cm}|p{1.5cm}|p{1.5cm}|}
%   \hline
%     \multicolumn{7}{ |c| }{Running Phone Based Expression with delay: 1000 ms} \\
%   \hline
% Sensor Name &Number of values &	Phone battery before &	Smart watch battery before &	Phone battery after	& Smart watch battery after &	Runtime (sec) \\
%   \hline
% movement 	& 	 \\
% gamerotation	 &  \\
% linear acceleration 	&  \\ 
% gravity 	&		 \\
% heartrate & 	 \\
%   \hline
%   \end{tabular}
% \end{center}
%% Appendix Template

\chapter{Merging SWAN Repositories} % Main appendix title

\label{AppendixB} % Change X to a consecutive letter; for referencing this appendix elsewhere, use \ref{AppendixX}

\lhead{Appendix B. \emph{Merging SWAN Repositories}} % Change X to a consecutive letter; this is for the header on each page - perhaps a shortened title
At the beginning of the master thesis project, the main repository of SWAN was holding the stable version of it, which was being used in production 
by a  company. Our SWAN project has a big team, each member work on different part of the SWAN or SWAN related functionality.
To avoid any breaking changes to the current SWAN version, we decided to make a different branch and put all my smartwatch related changes in it.
When we was done with implementing the first prototype of SWAN WEAR sensors, other PhD students' projects: SWAN Plus\cite{swan_plus} and SWAN-Fly\cite{swan_fly} were also ready.
Unfortunately, the commits of the projects stated above were in different repositories so merging them into master branch was proven to be difficult.
We followed this procedure when merging the SWAN:
\begin{itemize}
 \item Create the release branch with the old, stable SWAN
 \item Create development branches for SWAN Plus and SWAN-Fly
 \item Export commits from SWAN Plus and SWAN-Fly as patches
 \item Apply the patches to each development branch
 \item Merge Smartwatch code into Master
 \item Merge SWAN Plus into master and resolve conflicts
 \item Merge SWAN-Fly into master and resolve conflicts
\end{itemize}

The most challenging part was to apply patches from the two different repositories. Merging commits were the reason why some history was 
lost, and patches were hard to apply. This also proves why you should always use the rebase option instead of merging option when you push commits
to the remote. Atlassian article\cite{atlassian_merge_rebase} explains the main advantage of rebasing over using merging commits.
%\input{Appendices/AppendixC}

\addtocontents{toc}{\vspace{2em}} % Add a gap in the Contents, for aesthetics

\backmatter

%----------------------------------------------------------------------------------------
%	BIBLIOGRAPHY
%----------------------------------------------------------------------------------------

\label{Bibliography}

\lhead{\emph{Bibliography}} % Change the page header to say "Bibliography"

\bibliographystyle{abbrv} % Use the "apalike" BibTeX style for formatting the Bibliography

\bibliography{Bibliography} % The references (bibliography) information are stored in the file named "Bibliography.bib"

\end{document}